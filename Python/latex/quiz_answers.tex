\documentclass[12pt]{article}
\usepackage[margin=1in]{geometry}
\usepackage{xcolor}
\usepackage{enumitem}
\usepackage{titlesec}
\usepackage{lmodern}
\usepackage{hyperref}
\hypersetup{colorlinks=true, linkcolor=cyan, urlcolor=cyan}

\definecolor{correct}{RGB}{42,157,143} % Teal for correct answers

%\titleformat{\section}{\normalfont\Large\bfseries\color{cyan}}{}{0pt}{}
\usepackage{titlesec}

% Make section titles larger and increase spacing
\titleformat{\section}
  {\normalfont\LARGE\bfseries\color{cyan}}  % font size and color
  {}{0pt}{}                                 % no numbering or extra label

\titlespacing{\section}
  {0pt}{30pt plus 10pt minus 5pt}{15pt}     % {left}{before}{after}


\begin{document}

\title{\Huge 🌌 Galaxy Quiz -- Answer Sheet}
\author{}
\date{}
\maketitle

% ------------- Galaxy Clusters -------------
\section*{Galaxy Clusters}

\subsection*{Q1: Which of the following statements about galaxy clusters is true?}
\begin{enumerate}[label=\Alph*.]
    \item Clusters contain mostly spiral galaxies
    \item \textbf{\textcolor{correct}{Dark matter constitutes about 80--85\% of the mass in clusters ✅}}
    \item Clusters are typically less than 1 Mpc in diameter
    \item Clusters are fully virialized at all times
\end{enumerate}
\textit{Explanation: Dark matter constitutes a significant fraction of the mass in galaxy clusters.}

\bigskip

\subsection*{Q2: What is the significance of the Sunyaev-Zeldovich effect in studying galaxy clusters?}
\begin{enumerate}[label=\Alph*.]
    \item It measures the temperature of the cluster gas
    \item It detects the gravitational lensing of clusters
    \item \textbf{\textcolor{correct}{It provides a mass-weighted measure of hot gas in clusters ✅}}
    \item It identifies the number of galaxies in a cluster
\end{enumerate}
\textit{Explanation: The Sunyaev-Zeldovich effect is crucial for understanding the hot gas in galaxy clusters.}

\bigskip

\subsection*{Q3: How does the spatial distribution of galaxies in cD clusters differ from that in spiral-rich clusters?}
\begin{enumerate}[label=\Alph*.]
    \item cD clusters have a uniform distribution of galaxies
    \item Spiral-rich clusters have a smooth and circularly symmetric distribution
    \item \textbf{\textcolor{correct}{cD clusters show rapid density increase towards the center ✅}}
    \item Spiral-rich clusters have a higher concentration of elliptical galaxies
\end{enumerate}
\textit{Explanation: In cD clusters, the galaxy density increases towards the center, making them highly concentrated.}

% ------------- Density Wave Theory -------------
\section*{Density Wave Theory}

\subsection*{Q1: What is the primary classification scheme proposed by Hubble for galaxies?}
\begin{enumerate}[label=\Alph*.]
    \item The spiral classification
    \item \textbf{\textcolor{correct}{The tuning fork diagram ✅}}
    \item The elliptical classification
    \item The morphological sequence
\end{enumerate}
\textit{Explanation: Hubble proposed the tuning fork diagram to classify galaxies based on morphology.}

\bigskip

\subsection*{Q2: Which type of galaxy is characterized by having a rotating disk and a central bulge but lacks spiral arms?}
\begin{enumerate}[label=\Alph*.]
    \item Elliptical galaxies
    \item \textbf{\textcolor{correct}{Lenticular galaxies ✅}}
    \item Irregular galaxies
    \item Barred spiral galaxies
\end{enumerate}
\textit{Explanation: Lenticular galaxies have a disk and bulge but no spiral arms.}

\bigskip

\subsection*{Q3: What distinguishes Population I stars from Population II stars in galaxies?}
\begin{enumerate}[label=\Alph*.]
    \item Population I stars are older and redder
    \item Population I stars are found in the bulge of galaxies
    \item \textbf{\textcolor{correct}{Population I stars are young, hot stars associated with spiral arms ✅}}
    \item Population I stars are primarily found in elliptical galaxies
\end{enumerate}
\textit{Explanation: Population I stars are young and found in the spiral arms of galaxies.}

\bigskip

\subsection*{Q4: What is the significance of the density wave theory in understanding spiral galaxies?}
\begin{enumerate}[label=\Alph*.]
    \item It explains the formation of elliptical galaxies
    \item \textbf{\textcolor{correct}{It describes how stars in spiral arms are formed from gas clouds ✅}}
    \item It accounts for the random motion of stars in the bulge
    \item It explains the lack of star formation in lenticular galaxies
\end{enumerate}
\textit{Explanation: The density wave theory explains the spiral arm structure of galaxies.}

\bigskip

\subsection*{Q5: How do the properties of galaxies correlate with their classification in the Hubble sequence?}
\begin{enumerate}[label=\Alph*.]
    \item All properties correlate perfectly with Hubble type
    \item Only mass and luminosity correlate well with Hubble type
    \item \textbf{\textcolor{correct}{Many physical properties, such as star formation activity, correlate with morphology ✅}}
    \item There is no correlation between properties and Hubble type
\end{enumerate}
\textit{Explanation: Many properties like star formation activity correlate with the Hubble classification.}

% ------------- Elliptical and Dwarf Galaxies -------------
\section*{Elliptical and Dwarf Galaxies}

\subsection*{Q1: What is the modern view of elliptical galaxies regarding their composition?}
\begin{enumerate}[label=\Alph*.]
    \item They contain only old stars with no gas or dust
    \item They are composed of young stars and abundant gas
    \item \textbf{\textcolor{correct}{They may contain hot X-ray gas, dust, and cold gas ✅}}
    \item They are exclusively formed from mergers of dwarf galaxies
\end{enumerate}
\textit{Explanation: Elliptical galaxies may contain hot X-ray gas, dust, and cold gas.}

\bigskip

\subsection*{Q2: Which law describes the surface brightness profile of elliptical galaxies?}
\begin{enumerate}[label=\Alph*.]
    \item The Tully-Fisher relation
    \item \textbf{\textcolor{correct}{The de Vaucouleurs law ✅}}
    \item The Hubble law
    \item The Sersic profile
\end{enumerate}
\textit{Explanation: The de Vaucouleurs law describes the surface brightness profile of elliptical galaxies.}

\bigskip

\subsection*{Q3: What is the primary support mechanism for the structure of elliptical galaxies?}
\begin{enumerate}[label=\Alph*.]
    \item Rotation of stars
    \item \textbf{\textcolor{correct}{Random motions of stars ✅}}
    \item Gas pressure
    \item Magnetic fields
\end{enumerate}
\textit{Explanation: Elliptical galaxies are supported by random motions of stars.}

\bigskip

\subsection*{Q4: How does the metallicity of stars in elliptical galaxies typically vary with radius?}
\begin{enumerate}[label=\Alph*.]
    \item It is uniform throughout the galaxy
    \item It decreases towards the center
    \item \textbf{\textcolor{correct}{It increases towards the center ✅}}
    \item It is higher in the outer regions
\end{enumerate}
\textit{Explanation: Metallicity tends to increase towards the center of elliptical galaxies.}

\bigskip

\subsection*{Q5: What distinguishes dwarf elliptical galaxies (dE) from normal elliptical galaxies?}
\begin{enumerate}[label=\Alph*.]
    \item Dwarf ellipticals are larger and more luminous
    \item Dwarf ellipticals have a higher star formation rate
    \item \textbf{\textcolor{correct}{Dwarf ellipticals are dark matter dominated and follow different correlations ✅}}
    \item Dwarf ellipticals contain only young stars
\end{enumerate}
\textit{Explanation: Dwarf elliptical galaxies are often dark matter dominated and follow different correlations.}

% ------------- Galaxy Evolution -------------
\section*{Galaxy Evolution}

\subsection*{Q1: What is the primary mechanism by which galaxies evolve over time?}
\begin{enumerate}[label=\Alph*.]
    \item Stellar explosions
    \item \textbf{\textcolor{correct}{Hierarchical merging of structures ✅}}
    \item Constant star formation
    \item Isolation from other galaxies
\end{enumerate}
\textit{Explanation: Galaxies evolve primarily through hierarchical merging of structures.}

\bigskip

\subsection*{Q2: Which of the following timescales is associated with the lifetime of massive stars?}
\begin{enumerate}[label=\Alph*.]
    \item \textbf{\textcolor{correct}{\textasciitilde100 Myr ✅}}
    \item \textasciitilde1 Gyr
    \item \textasciitilde10 Gyr
    \item \textasciitilde10 Myr
\end{enumerate}
\textit{Explanation: Massive stars have lifetimes on the order of \textasciitilde100 million years.}

\bigskip

\subsection*{Q3: What observational evidence supports the evolution of galaxies?}
\begin{enumerate}[label=\Alph*.]
    \item The presence of dark matter
    \item \textbf{\textcolor{correct}{Stellar populations in the Milky Way ✅}}
    \item The uniformity of cosmic microwave background radiation
    \item The distribution of quasars
\end{enumerate}
\textit{Explanation: Stellar populations in galaxies provide important evidence for their evolution.}

\bigskip

\subsection*{Q4: What is the significance of the Butcher-Oemler effect in galaxy clusters?}
\begin{enumerate}[label=\Alph*.]
    \item It indicates the presence of dark matter
    \item \textbf{\textcolor{correct}{It shows that blue galaxies are more common at higher redshifts ✅}}
    \item It describes the merger rates of galaxies
    \item It explains the formation of elliptical galaxies
\end{enumerate}
\textit{Explanation: The Butcher-Oemler effect shows that blue galaxies are more common at higher redshifts.}

\bigskip

\subsection*{Q5: How do stellar population synthesis models contribute to our understanding of galaxy evolution?}
\begin{enumerate}[label=\Alph*.]
    \item \textbf{\textcolor{correct}{They predict the mass of dark matter halos ✅}}
    \item They estimate the number of galaxies in the universe
    \item They simulate the number of galaxies in the universe
    \item They measure the distance to galaxies
\end{enumerate}
\textit{Explanation: Stellar population synthesis models help in predicting the mass of dark matter halos.}

\end{document}
